\section{An Overview of the Proof}\label{sec:overview}
Before we dive into the details of this proof, I think it is helpful
to step back and summarize it at a high level. The proof using a
method called forcing\footnote{Forcing was in fact developed to solve
  this problem}. The basic premise of the proof is that we want to
construct a model of ZFC which forces a set much larger than $2^{N}$,
$B$, to have an injection into $2^{N}$ in the topos
$\sheaves[\neg\neg]{P}$ for some poset $P$. It doesn't particularly
matter what $B$ is so long as it's strictly larger than $2^N$. For
simplicity, let us fix
\[
  B = \pow{\pow{N}}
\]
for the remainder of the paper. Since this cannot be done ordinarily
we use forcing to gradually introduce fragments of such an injection
so that, internally to the topos, we may manipulate this arrow as if
it were a complete injection. The ``fragments'' of the function will
constitute the poset $P$ that we're defining sheaves over. This is the
essence of topos-theoretic forcing. This topos is developed in
section~\ref{sec:cohentopos}.

Now in our model, we then have
$\sheafify(\Delta N) \mono \sheafify(\Delta B) \mono \Omega_{\neg\neg}^N$.
Having done this, we can actually force an object to appear strictly
between $N$ and $\Omega_{\neg\neg}^N$. This object however isn't
$\sheafify(\Delta B)$!  Instead, we use $\sheafify(\Delta(2^N))$, the
powerset of the NNO from the presheaf topos included into
$\sheaves[\neg\neg]{P}$. Since $B$ was specifically chosen to be much
larger than $2^N$, we get an inclusion
\[
  \sheafify(N) \mono \sheafify(\Delta 2^N) \mono \sheafify(B) \mono \Omega_{\neg\neg}^N
\]
All of these inequalities are developed in
section~\ref{sec:cohentopos}. Moreover, those first two
inclusions are strict in $\presheaf{P}$. We then prove that posets
like $P$ satisfy what is called the Souslin property and that in this
case $\sheafify(-)$ preserves the strictness of inclusions. This is
the most technical portion of the proof and is developed
in~\ref{sec:strictness}.

Having done this, by transitivity we have then forced the existence of
an object $\sheafify(2^N)$ which lies between $N$ and $\pow{N}$ as
required. Now the astute reader might object that we have not actually
provided a model of ZFC. This is largely the construction for a model
of ZFC is easily found in~\citet{Fourman:80}.
