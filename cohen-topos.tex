\section{The Cohen Topos}\label{sec:cohentopos}
\subsection{The Partial Order \texorpdfstring{$P$}{P}}\label{subsec:partialorder}\hfill

For the remainder of this development, we will work with the Cohen
topos. This is a subtopos of presheaves on a particular partial
order. As mentioned above, we wish to design this partial order in
order to help us construct a monomorphism from $B \to \Omega^N$. Now
by transposition, such a morphism can always be represented as
$B \times N \to \Omega$. This gives us an indication for how to
construct $P$, it can simply be subsets of $B \times N$. Now in order
for a function $B \to \Omega^N$ (transposed as
$B \times N \to \Omega$) to be a monomorphism, it must be that for any
$p$ and any $b \neq b' \in B$ that there is some $n$ so that
$p(b, n) \neq p(b', n)$.

Therefore, our partial order $P$ shall be a collection
\[
  P = \powfin{\{f \in B \times N \times \{0, 1\} \mid f \text{ functional}\}}
\]
For any $p \in P$, we shall treat $p$ as a member of
$B \times N \pto{} \{0, 1\}$ which is defined on only a finite number of
inputs. Accordingly, we shall use $p(b, n) \downarrow$ and
$p(b, n)\uparrow$ to indicate whether or not $p$ is or isn't defined
on a particular input respectively. $P$ is called the collection of
forcing conditions and each element is thus a condition. It should be
thought of as a ``constraint'' on the map that we are trying to
construct from $B \to \Omega^N$. If we are working at forcing
condition $p$ we are in effect stating that while we do not know the
full contents of the map $B \to \Omega^N$, we know that it is at least
a completion of $p$.

The finiteness of each set is crucial. It is used to imply that each
completion \emph{could} be part of a monomorphism. This is because for
any $b, b' \in \dom(p)$ for some $p \in P$, there exists an $n$ so
that $p(b, n) \uparrow$ and $p(b', n) \uparrow$. Therefore, it is
always the case that a future condition may add some data to
distinguish $b$ and $b'$. It remains to define an order on $P$
however. Let us say
\[
  q \le p \triangleq \forall (b, n) \in B \times N.
  \ p(b, n) \downarrow \implies p(b, n) = q(b, n)
\]
This is the opposite of the traditional order that partial functions
are endowed with and is clearly reflexive, antisymmetric, and
transitive. This reversal is typical in forcing developments and may
seem slightly confusing. The reason for it though is quite straight
forward, each $p$ is thought of representing our knowledge about some
map from $B \to \Omega^N$. The more defined $p$ is, the smaller it is
according to our ordering, the fewer maps it corresponds to. The
ordering is in effect the traditional subset ordering on the
possibilities each condition allows. Hence, a larger condition has
more information, it permits fewer possibilities and is therefore
smaller.

\subsection{Sheaves on \texorpdfstring{$P$}{P}}\label{subsec:presheaves}\hfill

Let us now turn our attention to the Cohen Topos, or
$\sheaves[\neg\neg]{P}$. We want to develop the framework necessary
for us to construct our cardinal inequalities that we hinted at in the
overview~\ref{sec:overview}. Before we can turn our attention to that
though, we would like to establish that our topos is a boolean topos
that satisfies the axiom of choice with a natural number object. Since
$\sheaves[\neg\neg]{P}$ is a subtopos of $\presheaf{P}$, it inherits
its NNO from $\presheaf{P}$. We know from
lemma~\ref{lem:background:notnotboolean} that $\sheaves[\neg\neg]{P}$
is boolean. In order to show that it satisfies the axiom of choice, we
would like to apply
lemma~\ref{lem:background:completealgebratoaoc}. In order to do this,
we must show that for each $F \in \sheaves[\neg\neg]{P}$ that
$\sub(F)$ is a complete boolean Heyting algebra, but this follows from
the fact that $\sheaves[\neg\neg]{P}$ is a Grothendieck topos and
lemma~\ref{lem:background:boolean}. This means all we must show is

\begin{lem}
  Subobjects of $1$ generate $\sheaves[\neg\neg]{P}$.
\end{lem}
\begin{proof}
  In order to show this, we will show that $\sheafify(\yoneda(p))$
  generates $P$ and that these sheaves are subobjects of one. First,
  the latter. Since $\yoneda(p) \mono 1$ clearly holds in $\presheaf{P}$, it
  suffices to note that, since $\sheafify$ is left exact
  $\sheafify(1) = 1$ and it preserves monomorphisms. Therefore,
  $\sheafify(\yoneda(p)) \mono \sheafify(1) = 1$ as required.

  Now we must show that $\sheafify(\yoneda(p))$ generates. It suffices
  to show that $\yoneda(p)$ generates $\presheaf{P}$. To see this,
  suppose that $f \neq g : F \to G$ in $\sheaves[\neg\neg]{P}$. In
  this case, we know that there is a $p$ and $t : \yoneda(p) \to F$ so
  that $ft \neq gt$. Therefore, $\sheafify(ft) \neq \sheafify(gt)$ but
  $\sheafify(ft) = f\sheafify(t) : \sheafify(\yoneda(p)) \to F$ and
  likewise for $gt$. Now to see that $y(p)$ generates it suffices to
  note that any presheaf is the colimit of representables. Therefore,
  if $f \neq g : P_1 \to P_2$ then this implies that there are two
  distinct cones on $P_2$ by the universality of a colimit. Therefore,
  one of the legs of the cones is different. Simply factoring that leg
  through $P_1$ gives our desired distinguishing map.
\end{proof}


Now that we have established that this topos is of the variety we
need, let us investigate what sheaves are actually found in it. First
we will demonstrate that the dense topology is
\emph{subcanonical}. This means that all yoneda embeddings form
sheaves, something that will prove useful later on.

\begin{lem}\label{lem:cohentopos:subcanonical}
  For any $p \in P$, $\yoneda(p)$ is a sheaf.
\end{lem}
\begin{proof}
  In order to do this, we will directly use the definition of sieve on
  Grothendieck topology. Suppose we have $S$, a covering sieve for the
  dense topology on $q$. Furthermore, suppose we have a matching
  family for $S$, $(x_f)_{f \in S}$. We now wish to find a unique
  amalgamation $x \in \yoneda(p)(q)$. Since $P$ is a poset, it
  suffices to show that such an amalgamation exists, as $y(p)(q)$
  contains either one or zero elements. Now, since $x_d \in y(p)(d)$
  for each $d \le q \in S$, we know that $d \le p$.

  All we need to show then is that $q \le p$. Suppose not. Then there
  must be a $(b, n)$ so that $q(b, n) \neq p(b, n)$ or that
  $q(b, n) \uparrow$.  Let us define
  $q' = q[(b, n) \mapsto \neg p(b, n)]$ so that it is clear that
  $q'(b, n) \neq p(b, n)$. Then, since $S$ is a cover in the dense
  topology and $q' \le q$ there must be a $d \le q'$ so that
  $d \in S$. However, we then have $d \le p$ by assumption so
  $d(b, n) = p(b, n)$, a contradiction. Therefore, it must be that
  $q \le p$ holds.
\end{proof}

Having fleshed out a few basic sheaves, let us now begin to construct
$\sheafify(\Delta B)$ and the monomorphism from it to
$\Omega_{\neg\neg}^{\sheafify(\Delta N)}$. Here we use $\Delta X$ to
indicate the constant presheaf $A \mapsto X$. First, we note that in
order to construct a morphism
\[
  g : \sheafify(\Delta B) \times \sheafify(\Delta N) \to \Omega_{\neg\neg}
\]
In order to construct this, we will look to construct a similar
morphism in $\presheaf{P}$. Let us define a subobject of
$A \mono \Delta B \times \Delta N$. First we note that
$\Delta B \times \Delta N = \Delta (B \times N)$. Therefore, we can
define $A$ as
\[
  A(p) = \{(b, n) \mid p(b, n) = 1\}
\]
This subobject is designed to pick out the graph of the function we
have currently available to us. It, crucially, will vary as we move
from condition to condition allowing $A$ to better and better
approximate our hypothetical monomorphism. Moreover, $A$ happens to be
a closed object of $\Delta B \times \Delta N$.

\begin{lem}\label{lem:cohentopos:Aclosed}
  $A$ is a closed subobject of $\Delta B \times \Delta N$ under
  $\neg\neg$.
\end{lem}
\begin{proof}
  It suffices to show that $\neg\neg A \le A$ as the reverse holds for
  any subobject. Suppose that $(b, n) \in (\neg\neg A)(p)$. This
  indicates that for all $q \le p$ there is an $r \le q$ for which we
  have $(b, n) \in A(r)$ which is to say, $r(b, n) = 1$. This in
  particular implies that $r(b, n) = 1$.

  Now if $(b, n) \not\in A(p)$. This implies that either $p(b, n) = 0$
  or that $p(b, n) \uparrow$. If the former is the case then we have
  an immediate contradiction: $r \le p$ so $p(b, n) = r(b, n)$ must
  hold. If the latter, then we have $p' = p[(b, n) \mapsto 0] \le p$
  and for all $r \le p'$, we know that $(b, n) \not\in A(r)$. This
  contradicts the assumption that $(b, n) \in (\neg\neg A)(p)$ so
  we're done.
\end{proof}

Having established that $A$ is a closed subobject, we know that
$\charmap(A) : \Delta B \times \Delta N \to \Omega$ must factor
through $\Omega_{\neg\neg} \mono \Omega$ by lemma~\ref{lem:background:closedfactors}.
Now this means we have the following diagram
\[
  \begin{tikzcd}
    {\Delta N \times \Delta B} \ar[rr, "\charmap(A)"]
    \ar[dr, "f", swap, dashed] & & \Omega\\
    & \Omega_{\neg\neg} \ar[ur, rightarrowtail]
  \end{tikzcd}
\]
Now we're going to show that this map
$g = \Lambda f : \Delta B \to \Omega_{\neg\neg}^{\Delta N}$ is
actually the desired monomorphism.
\begin{lem}
  $g$ is a monomorphism in $\presheaf{P}$.
\end{lem}
\begin{proof}
  It suffices to show that this is a monomorphism in $\cat{Set}$ for
  all $p \in P$. Now unfolding definitions, we know that
  \[
    g_p : B \to (\Delta N \times \yoneda(p) \to \Omega_{\neg\neg})
  \]
  Moreover, if we apply $g_p$ to $b \in B$ and
  $(n, \star) : \Delta(N)(q) \times \yoneda(p)(q)$ we get
  \[
    (g_p(b))_q(n, \star) = \{r \mid r \le q \wedge r(b, n) = 1\}
  \]
  Suppose that $b \neq c \in B$, we wish to show that there is an $n$
  so that
  \[
    (g_p(b))_q(n, \star) \neq (g_p(c))_q(n, \star)
  \]
  However, since $q$ is finite, there is an $n$ so that
  $q(b, n_0) \uparrow$ and $q(c, n_0) \uparrow$. We then consider
  $(g_p(b))_q(n_0, \star)$ and $(g_p(c))_q(n_0, \star)$. Since
  $q(b, n_0) \uparrow$, $(g_p(b))_q(n_0, \star)$ must contain $q'$
  where $q' = q[(b, n_0 \mapsto 1)]$. However, since $q'(c, n_0)
  \uparrow$ it cannot be that $q' \in (g_p(c))_q(n_0, \star)$ so these
  are distinct as required. Therefore, $g$ is a mono.
\end{proof}

\begin{cor}
  $m = \sheafify(g)$ is a monomorphism in $\sheaves[\neg\neg]{P}$ from
  $\sheafify(\Delta B) \to \Omega_{\neg\neg}^{\sheafify(\Delta N)}$.
\end{cor}
\begin{proof}
  Since $\sheafify$ is left exact, we immediately have that $m$ is a
  mono. It remains to show that
  \[
    \sheafify(\Omega_{\neg\neg}^{\Delta N}) \cong
    \Omega_{\neg\neg}^{\sheafify(\Delta N)}
  \]
  This follows because
  \begin{align*}
    \hom(X, \Omega_{\neg\neg}^{\Delta N}) &=
        \hom(\Delta N \times X, \Omega_{\neg\neg})\\
     &= \hom(\sheafify(\Delta N \times X), \Omega_{\neg\neg})\\
     &= \hom(\sheafify(\Delta N) \times \sheafify(X), \Omega_{\neg\neg})\\
     &= \hom(\sheafify(X), \Omega_{\neg\neg}^{\sheafify(\Delta N) })\\
     &= \hom(X, \Omega_{\neg\neg}^{\sheafify(\Delta N) })\\
  \end{align*}
  so this follows immediately from yoneda. Above we have made use of
  the fact that
  \[
    \hom(X, Y) = \hom(\sheafify(X), Y)
  \]
  when $Y$ is a sheaf, an immediate result of the fact that sheaves
  form a reflective subcategory with $\sheafify$.
\end{proof}

Now let us stop and take stock. At this point, we have the
\[
  \begin{tikzcd}
    \sheafify(\Delta N) \ar[r, rightarrowtail, "\sheafify(\Delta(\iota))"] &
    \sheafify(\Delta B)  \ar[r, rightarrowtail, "m"] &
    \Omega_{\neg\neg}^{\sheafify(\Delta B)}
  \end{tikzcd}
\]
So we've completed the task of ``forcing''
$\Omega_{\neg\neg}^{\sheafify(\Delta N)}$ to become quite large. In
fact we can insert $\sheafify(\Delta 2^N)$ into the above diagram
just using the monomorphism we have present in $\presheaf{P}$. Let us
factor $\iota : N \to B$ as $\iota_1 : N \to 2^N$ and
$\iota_2 : 2^N \to B$. Then,
\[
  \begin{tikzcd}
    \sheafify(\Delta N) \ar[r, rightarrowtail, "\sheafify(\Delta(\iota_1))"] &
    \sheafify(\Delta 2^N) \ar[r, rightarrowtail, "\sheafify(\Delta(\iota_2))"] &
    \sheafify(\Delta B) \ar[r, rightarrowtail, "m"] &
    \Omega_{\neg\neg}^{\sheafify(\Delta B)}
  \end{tikzcd}
\]
What remains then is to prove that no new epimorphisms have been
introduced in this new topos and that all these first two inclusions
are still strict.
