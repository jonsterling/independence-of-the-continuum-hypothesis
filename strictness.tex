\section{The Preservation of Strictness}\label{sec:strictness}

Proving the strictness of the cardinal inequalities turns out to be a
rather technical endeavor. To begin with, we need to define what it
means categorically for a monomorphism to be strict. In order to do
this, we will use the internal language of the topos to define a
subobject $\epis(X, Y) \mono Y^X$.
\[
  \epis(X, Y) = \{f \mid \forall y \in Y.\ \exists x \in X.\ f(x) = y\}
\]
First we show that this actually corresponds to the epimorphisms from
$X$ to $Y$. For this we will make heavy use of the Kripke-Joyel
semantics explained in~\citet[Chapter~6]{MacLane:92}.

\begin{lem}\label{lem:strictness:epis}
  $\langle \pi_1, e \rangle : P \times X \epi P \times Y$ if and only if
  $\Lambda e : P \to Y^X$ factors through $\epis(X, Y)$.
\end{lem}
\begin{proof}
  It suffices to show that $P \Vdash \forall x.\ \exists y.\ f(x) = y$
  if and only if $\hat{f} : P \times X \to Y$ is epi. Now by the
  Kripke-Joyal semantics, $P \Vdash \forall x.\ \exists y.\ f(x) = y$
  if and only if for all $V$,
  \[
    P \times V \Vdash \exists y.\ (f\pi_1)(x) = \pi_2
  \]
  Next, this holds if and only if for some $p : U \epi P \times V$
  and some $b : U \to X$
  \[
    U \Vdash f\pi_1p(b) = \pi_2p
  \]
  Now the rule for equality simply states that this holds if the
  interpretation of these two maps are equal. That is, that
  $\epsilon \circ \langle f\pi_1p, b \rangle = \pi_2p$. However, then
  we know that
  \[
    \epsilon \circ \langle f\pi_1p, b \rangle =
    \hat{f} \circ \langle \pi_1 p, b \rangle
  \]
  here $\hat{f}$ is just the transpose of $f$. Finally, this gives us that
  $p = \langle \pi_1, \hat{f} \rangle \circ \langle \pi_1 p, b \rangle$.
  This tell us that $\langle \pi_1, \hat{f} \rangle$ is epi and this is
  true if and only if $\hat{f}$ is epi as required.
\end{proof}

Now that we have established an internal representation of the
existence of epimormorphisms we can define strict inequality.

\begin{defn}
  $X < Y$ if and only if $X \mono Y$ and $\epis(X, Y) \cong 0$.
\end{defn}

Having internalized this, we are at least now in a position to state
the theorem that we want to prove:
$X < Y \implies \sheafify(\Delta X) < \sheafify(\Delta Y)$. This
property will rely crucially on the structure of $P$. In particular we
shall show that $P$ has the Souslin property.

\begin{defn}\label{defn:strictness:souslin}
  A partial order $Q$ satisfies the Souslin property if any set of
  objects which are pairwise disjoint ($a \wedge b = 0$ for all $a$
  and $b$) is at most countable.
\end{defn}
\begin{defn}\label{defn:strictness:souslintopos}
  A topos $\Etop$ satisfies the Souslin property if it is generated by
  objects for whom $\sub(-)$ has satisfies the Souslin
  property~\ref{defn:strictness:souslin}
\end{defn}

In fact the Souslin property is precisely what we need in order to get
this fact to go through as the following lemma shows.

\begin{lem}\label{lem:strictness:souslin}
  If $X < Y$ in $\cat{Set}$ and $X$ and $Y$ are infinite. Grothendieck
  topos $\Etop$ which satisfies the Souslin property then
  $\sheafify(\Delta X) < \sheafify(\Delta Y)$.
\end{lem}
\begin{proof}
  It is clear that if $X \le Y$, then
  $\sheafify(\Delta X) \le \sheafify(\Delta Y)$. Therefore, it
  suffices to show that if $\epis(X, Y) = 0$ then
  $\epis(\sheafify(\Delta X), \sheafify(\Delta Y)) = 0$ as well. Let
  us suppose not. Then there must be a nonzero object $U$ which
  satisfies the Souslin property so that $U \to \epis(X, Y)$.
  Therefore, by lemma~\ref{lem:strictness:epis} there must be an
  epimorphism
  $g = \langle \pi_1, f \rangle : U \times X \epi U \times Y$.

  Take $x \in X$ and $y \in Y$, have two points
  $\sheafify(\Delta x) : 1 \to \sheafify(\Delta X)$ and
  $\sheafify(\Delta y) : 1 \to \sheafify(\Delta Y)$. Using these, we
  can form the two pullback squares
  \[
    \begin{tikzcd}
      V_{x, y} \ar[d, rightarrowtail] \ar[r, rightarrowtail] &
      P_y \ar[d, rightarrowtail] \ar[r, "h", twoheadrightarrow] &
      {U \cong U \times 1} \ar[d, "{(1, \sheafify(\Delta y))}"]\\
      {U \cong U \times 1} \ar[r, "{(1, \sheafify(\Delta x))}", swap]&
      U \times \sheafify(\Delta X) \ar[r, "g", twoheadrightarrow, swap]&
      U \times \sheafify(\Delta Y)
    \end{tikzcd}
  \]
  Now let us define
  $W = \{(x, y) \in X \times Y \mid V_{x, y} \neq 0\}$. First note
  that $S \cong \amalg_{x \in X} 1$. Therefore,
  $\amalg_{x \in X} 1 \times U \cong \sheafify(\Delta X) \times U$.
  Moreover, this colimit exists because $\Etop$ is a Grothendieck
  topos. Since pullbacks have a right adjoint, we know that pulling
  back along $P_y \to U \times \sheafify(\Delta X)$ gives us that
  $\amalg_x V_{x, y} \to P_y$. However, since
  $\amalg_x U \times 1 \cong U \times \sheafify(\Delta X)$ we know
  that this isomorphism is preserved by pullback so in fact
  $\amalg_x V_{x, y} \cong P_y$. Now since $U$ is known to be nonzero
  and we have an epimorphism $P_y \epi U$, it must be that $P_y$ is
  also nonzero. Therefore, we know that there is some $x, y$ so that
  $V_{x, y}$ is nonempty since $\amalg_x V_{x, y} \cong P_y$. This
  tells us for every $y \in Y$ there exists an $x \in X$ so that
  $(x, y) \in W$.

  This tells us that $\pi_2 : W \to Y$ is a surjection of sets so it
  suffices to show that $X \epi W$ in order to show our contradiction
  that $X \epi Y$. Now, in order to this, let us first note that
  $V_{x, y} \mono U$. Moreover, if $y \neq y'$, then it must be that
  $V_{x, y} \wedge V_{x, y'} = 0$. This is because $y : 1 \mono Y$ and
  $y' : 1 \mono Y$ clearly are disjoint subobjects of $Y$ in
  $\cat{Set}$. However, this pullback diagram is preserved by $\Delta$
  since limits are computed pointwise and then by $\sheafify$ since it
  is left exact. Therefore,
  $\sheafify(\Delta(y)) \wedge \sheafify(\Delta(y')) = 0$. Finally,
  since meets are preserved by pullback and so is $0$, this gives us
  that $V_{x, y} \wedge V_{x, y'} = 0$.

  Now it is time to make use of this Souslin property. We know that
  $W_x = \{y \mid (x, y) \in W\}$ must be at most countable as they
  are necessarily disjoint. Since $S$ is assumed to be infinite, since
  we know that
  \[
    \left\vert W \right\vert = \left\vert S \right\vert \times \omega
    = \left\vert S \right\vert
  \]
  Therefore, $S \cong W$ and we have our desired surjection.
\end{proof}

Having proving all of this, all that remains is to show that our $P$
does in fact satisfy the Souslin property. This turns out to be a
fun\footnote{Boring.} exercise in order theory.

\begin{lem}
  $\sheaves[\neg\neg]{P}$ satisfies the Souslin property.
\end{lem}
\begin{proof}
  We know that $\yoneda(p)$ generates $\sheaves[\neg\neg]{P}$. It is
  clear that if $A \mono B$ and $B$ has the Souslin property, then so
  does $A$ (as $\sub(A) \subseteq \sub(B)$). Therefore, since
  $\yoneda(p) \mono 1$, it suffices to show that $1$ has the Souslin
  property. Therefore, suppose that $(U_i)_i$ is a family of pairwise
  disjoint nonzero subterminals. We wish to show that it is at most
  countable. Now we know that $\yoneda(p_i) \le U_i$ and that
  $U_i \wedge U_j = 0$ so $\yoneda(p_i) \wedge \yoneda(p_j) =
  0$. Therefore, $(U_i)_i$ really represents a set of pairwise
  incompatible conditions. We wish to show that this is at most
  countable.

  Let us define
  \[
    A_i = \{p_i \mid p_i \text{ is defined for $n$ entries}\}
  \]
  We wish to show by induction on $i$ that each $A_i$ is
  countable. Since $\bigcup_i A_i = (p_i)_i$ this shows our original
  goal. Suppose that $A_i$ is countable for all $i < j$, we wish to
  show that $A_j$ is countable. To show that $A_j$ is countable, it
  suffices to show that $A_{j, n}$ is countable where
  \[
    A_{j, n} = \{p_i \mid p_i \in A_j \wedge \exists b.\ p_i(b, n) \downarrow\}
  \]
  this is because $\bigcup_{i \in \mathbb{N}} A_{j, i} = A_j$. Now we
  can divide each $A_{j, n}$ into two sets, $A_{j, n, 0}$ and $A_{j, n
    1}$ where
  \[
    A_{j, n, i } = \{p_i \mid p_i \in A_j \wedge \exists b.\ p_i(b, n) = i\}
  \]
  However, we note that  $A_{j, n, i}$ must be comprised of pairwise
  incompatible counditions still. Since we know that for any $p, q \in
  A_{j, n i}$ that $p(b_p, n) = q(q_p, n)$, it must be that there is
  some other $b', n'$ so that $p(b', n') \neq q(b', n')$. Therefore,
  the set
  \[
    R_{j, n, i } = \{p_i \setminus \{(b_{p_i}, n, i)\} \mid p_i \in A_{j, n, i}\}
  \]
  is pairwise incompatible. Since it is comprised of conditions of
  length $j - 1$, it must be that $R_{j, n, i} \subseteq A_{j - 1}$ so
  it is countable. Furthermore, then $A_{j, n, i}$ is countable and so
  is $A_{j, n}$ as we required. Therefore, $A_j$ is countable and we
  are done by induction.
\end{proof}
Now all told, this gives us that there is no epimorphism
$\sheafify(\Delta \Omega^N) \epi \sheafify(\Delta B)$ nor an
epimorphism $\sheafify(\Delta N) \epi \sheafify(\Delta \Omega^N)$ so
that
\[
  \begin{tikzcd}
    \sheafify(\Delta N) \ar[r, rightarrowtail, "\sheafify(\Delta(\iota_1))"] &
    \sheafify(\Delta \Omega^N) \ar[r, rightarrowtail, "\sheafify(\Delta(\iota_2))"] &
    \sheafify(\Delta B) \ar[r, rightarrowtail, "m"] &
    \Omega_{\neg\neg}^{\sheafify(\Delta B)}
  \end{tikzcd}
\]
indeed has strict inclusions for the first two maps. Thus, we have
established an object which lies strictly between
$\sheafify(\Delta N)$ and $\Omega_{\neg\neg}^{\sheafify(\Delta N)}$.
This, combined with the result of~\citet{Fourman:80} is sufficient to
establish the independence of the continuum hypothesis from ZFC.
